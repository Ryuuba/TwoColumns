\usepackage[cm]{fullpage}
\usepackage{listingsutf8} % Embebbed formatted code to the file
\lstset{basicstyle=\footnotesize\ttfamily,language=C++,keywordstyle=\color{Xoc},stringstyle=\color{Izt},}
\usepackage[protrusion=true,expansion=true]{microtype} % Better typography
\usepackage{graphicx} % Required for including pictures
\graphicspath{{Figures/}{Logo/}} % Directories where images are stored
\usepackage{wrapfig} % Allows in-line images
\usepackage[svgnames]{xcolor} % Allows the use of colors across the document
   \definecolor{Azc}{RGB}{205,3,46}
   \definecolor{Izt}{RGB}{87,165,25}
   \definecolor{Xoc}{RGB}{0,114,206}
   \definecolor{Cua}{RGB}{240,130,0}
   \definecolor{Ler}{RGB}{173,37,168}
   \definecolor{Wolf}{RGB}{64,64,64}
\usepackage{amsfonts, amsmath, amsthm, amssymb}
\usepackage{sectsty}
\chapterfont{\color{Xoc}}
\sectionfont{\color{Xoc}}
\subsectionfont{\color{Xoc}}
\usepackage{hyperref}
\hypersetup{
   colorlinks=true,
   linkcolor=black,
   citecolor=black,
   linkbordercolor=white,
   urlcolor=Xoc,
   citecolor=Xoc
}
\usepackage{url} % Become references in links
\usepackage[spanish,mexico]{babel} % Requiered for Mexican Spanish hyphenation and table names
\usepackage[utf8x]{inputenc} % Required for accented characters
\usepackage[T1]{fontenc}
\usepackage{mathpazo} % Use the Palatino font
\usepackage{algorithm,algpseudocode} %required to write pseudocode
%\usepackage[sort&compress]{natbib} %change a list of citation [1], [2], [3] to [1-3]
\usepackage{siunitx} %Requiered to use SI units
\usepackage{booktabs} %Fancy tables
\usepackage[font=footnotesize,labelfont=bf]{caption}
\captionsetup[algorithm]{font=footnotesize}
\usepackage{abstract}
\setlength{\absleftindent}{0mm}
\setlength{\absrightindent}{0mm}
\setlength{\parindent}{0mm} % Do not indent the 1st line of a paragraph.
\setlength{\parskip}{3mm}   % Add space between paragraphs.


\linespread{1.05} % Changes line spacing here, Palatino benefits from a slight increase by default

\makeatletter
\renewcommand\@biblabel[1]{\textbf{#1.}} % Changes the square brackets for each bibliography item from '[1]' to '1.'
\renewcommand{\@listI}{\itemsep=0pt} % Reduces the space between items in the itemize and enumerate environments and the bibliography

\renewcommand{\maketitle}{ % Customizes the title - do not edit title and author name here, see the TITLE block below
\begin{center} % Left align
{\LARGE\@title} % Increases the font size of the title

\vspace{25pt} % Some vertical space between the title and author name

{\small\@author} % Author name
\\\@date % Date

\vspace{15pt} % Some vertical space between the author block and abstract
\end{center}
}

\renewcommand{\abstractname}{Resumen} % Uncomment to change the name of the abstract to something else
\renewcommand{\lstlistingname}{Código} % Listing -> Code
\renewcommand{\ALG@name}{Algoritmo}
\newcommand{\white}[1]{\textcolor{white}{#1}}
\newcommand{\red}[1]{\textcolor{Azc}{#1}}
\newcommand{\blue}[1]{\textcolor{Xoc}{#1}}
\newcommand{\orange}[1]{\textcolor{Cua}{#1}}
\newcommand{\grape}[1]{\textcolor{Ler}{#1}}
\newcommand{\green}[1]{\textcolor{Izt}{#1}}
\newcommand{\eg}{\textit{e.g. }}
\newcommand{\ie}{\textit{i.e. }}
